\section*{Abstract}

The abstract corresponds to a condensed, yet complete version of the entire research work or project. It includes the problem statement, methodology, main findings, and conclusions. Its primary goal is to provide readers with a clear and concise view of the work's content, allowing them to quickly determine its relevance to their own interests.

First, one should begin with a brief description of the research question or problem to be solved that the work addresses. This sets the stage for the reader, highlighting the importance of the research and its potential contribution to the field. It is essential to ensure that this statement is specific and directly related to the study's scope. This part of the abstract concludes by clearly establishing the overall goal of the work.

Next, the methods that were used to conduct the research should be summarized. This part should cover the research design, data collection techniques, and analysis methods. This description should be kept brief, but informative enough to allow readers to understand how the research was conducted, without delving into excessive detail.

Then, the abstract should focus on the main results or findings of the research. These findings should be presented briefly, emphasizing the most significant data and outcomes. It is necessary to avoid including all results; instead, select those that most directly address the research question or problem to be solved, and that have the most significant implications to the general context.

Finally, the most significant conclusions drawn from the results are included, along with their implications for the state of the art and potential future research. This section should briefly articulate the importance of the work and how it advances knowledge or understanding within your discipline.

An abstract for a thesis or dissertation should be concise, typically no more than one page in length. It should be written in a clear and direct manner, avoiding jargon and complex language to ensure accessibility to a broad audience. Preferably use active voice and past tense when discussing the research conducted, although present tense may be appropriate for the implications and conclusions if they extend beyond the scope of the completed work.



