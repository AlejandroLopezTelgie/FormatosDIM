\section{Introducción}

\subsection{Contexto}
El contexto, abarcando ya sea la pregunta de investigación o la declaración del problema, sirve como la base de una memoria o tesis. Se define el alcance, la importancia y la orientación de la investigación, guiando todo el estudio. Crear un contexto convincente y claro es crucial, ya que no solo informa a los lectores sobre el propósito de la investigación, sino que también establece la necesidad y relevancia de la investigación.

Se comienza resumiendo el estado actual del conocimiento en su campo de estudio (estado del arte). Aquí, se destacan las investigaciones existentes y se señalan las brechas o preguntas sin responder que el estudio pretende abordar. Este enfoque no solo prepara el escenario para la investigación, sino que también demuestra su necesidad y contribución potencial al campo.

Después de establecer la brecha, se articula la pregunta de investigación o declaración del problema a resolver según sea el caso. Esto debería abordar directamente la brecha identificada, especificando lo que la investigación investigará o resolverá. Uno debe asegurarse que la pregunta o problema sea enfocado, investigable y significativo para el campo de estudio.

Luego, se debe explicar por qué la pregunta de investigación o problema es importante de abordar. Se discuten las implicaciones potenciales de su estudio y cómo podría contribuir al campo o aplicarse a problemas del mundo real. Esta justificación establece el valor y la contribución del estudio. Poner cuidado en que la pregunta de investigación o declaración del problema sea lo suficientemente específica para ser manejable dentro del alcance de la memoria o tesis. 

Al elaborar el contexto, el foco es involucrar a los lectores presentando una narrativa clara y convincente que destaque la importancia de su investigación y cómo llena una brecha específica en el estado del arte. Este elemento fundamental establece el tono para todo el estudio, guiando la dirección de la investigación e informando su metodología y análisis.

\subsection{Hipótesis}
La hipótesis de un estudio es una declaración clara y comprobable que predice un resultado basado en la comprensión iniciañ del problema de investigación. Debe ser formulada después de una revisión exhaustiva de la literatura y la identificación de brechas dentro de ella. Una buena hipótesis se relaciona directamente con la pregunta de investigación y está enmarcada de manera que permite su comprobación mediante investigación empírica. Típicamente adopta una forma declarativa, postulando una relación entre variables que puede ser respaldada o refutada por los datos recopilados (establecimiento de hipótesis nula). Se debe asegurar claridad y especificidad en la hipótesis para facilitar un diseño de investigación y análisis enfocado, convirtiéndola en un elemento importante del estudio.

\subsection{Objetivos}
Los objetivos de una memoria o tesis describen lo que la investigación pretende lograr. Se divide en objetivo general, que proporciona una visión global del objetivo de la investigación, y objetivos específicos, que desglosan el objetivo general en resultados más pequeños y medibles. Escribir objetivos claros y concisos es esencial para guiar el proceso de investigación y clarificar el enfoque del estudio tanto para el investigador como para los lectores.

Se comienza con el objetivo general expresando la meta global de la investigación. Esto debe ser una declaración concisa que refleje el objetivo general del estudio, derivado directamente de la declaración del problema o de la pregunta de investigación. Dirige el resultado esperado a un alto nivel, como entender un fenómeno, determinar la efectividad de una estrategia o explorar una relación entre variables.

Los objetivos específicos deben desglosar el objetivo general en componentes más pequeños y detallados. Cada objetivo específico se centra en un aspecto diferente de la investigación, contribuyendo colectivamente a lograr el objetivo general. Deben ser declaraciones claras, precisas, realizables y medibles que guíen la metodología y el análisis. Que cada objetivo específico sea medible, significa que hay un criterio claro de éxito. Esto hace que los resultados de la investigación sean verificables y permite un enfoque sistemático para la recolección y análisis de datos.

Uno debe asegurarse que sus objetivos específicos se alineen con los componentes de la investigación, como la revisión de literatura, metodología, análisis de datos y discusión. Cada objetivo debe conducir a una sección de su investigación, creando una estructura coherente que aborde sistemáticamente el problema de investigación o pregunta a resolver.


\subsection{Descripción del trabajo}
Escribir la descripción del trabajo en relación con los objetivos implica vincular claramente las actividades de investigación y contenido con los objetivos específicos delineados. Esta sección demuestra cómo cada parte del estudio contribuye a alcanzar los objetivos específicos, asegurando coherencia y enfoque a lo largo del trabajo.

Se debe comenzar detallando las actividades de investigación, métodos y enfoques adoptados. Cada actividad se conecta explícitamente con uno o más de los objetivos específicos. Esto muestra cómo cada elemento del trabajo está diseñado intencionalmente para contribuir a alcanzar las metas.

Es importante organizar esta descripción en una secuencia lógica que refleje el flujo del proceso de investigación. Se comienza con pasos preliminares como la revisión de literatura, seguido por la metodología, recolección de datos, análisis y, finalmente, discusión y conclusión. Para cada paso, se explica cómo se abordan los objetivos específicos, contribuyendo al objetivo general.


