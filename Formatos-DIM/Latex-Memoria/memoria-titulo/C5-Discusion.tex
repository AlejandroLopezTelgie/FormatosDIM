\section{Discusión}

La sección de discusión de una memoria o tesis es donde se interpretan los hallazgos, se colocan dentro del contexto más amplio del estado del arte y se exploran sus implicaciones. Es una parte fundamental del trabajo que se enfoca no solo en analizar críticamente la propia investigación, sino que también en conectarla con el conocimiento existente.

Se suele comenzar interpretando los resultados de la investigación. Se discute lo que los hallazgos significan en relación con las preguntas de investigación o problema a resolver. Es importante ir más allá de simplemente reiterar los resultados; analizar su significancia, patrones y cualquier anomalía. En caso de haber elaborado una hipótesis inicial, en esta sección se discute la validez o rechazo de ella.

Un aspecto clave de la interpretación de los hallazgos es su nivel de confianza y el lenguaje asociado a ello. Para resultados muy categóricos, se suelen utilizar verbos como: demostrar, probar, acreditar o verificar. A medida que disminuye la confianza en los resultados, ya sea por una metodología deficiente o por que los datos no son concluyentes, se usan verbos como: indicar, mostrar, sugerir o señalar.

Luego, se ubican los hallazgos en el contexto de la literatura más amplia. Se pueden comparar y contrastar los resultados con estudios previos, teorías o modelos del estado del arte. Se debe resaltar cualquier similitud o diferencia, y discutir por qué podrían existir. Esta comparación ayuda a validar la investigación y la sitúa dentro del estado del arte.

Todo estudio tiene sus limitaciones, y es esencial discutirlas abiertamente. Es importante reflexionar sobre las limitaciones de la metodología, datos y análisis, y considerar cómo podrían afectar la interpretación de los hallazgos. Así, se busca elaborar sobre la generalización o no de los hallazgos. Basado en los hallazgos y las limitaciones identificadas, se pueden sugerir lineamientos para investigaciones futuras. Se pueden destacar preguntas sin responder y proponer maneras en las cuales estudios futuros podrían construir sobre el trabajo.

Las secciones de discusión y conclusiones cumplen propósitos complementarios, pero distintos. La discusión es donde los hallazgos se analizan profundamente, se contextualizan dentro de la literatura existente y se exploran por sus implicaciones más amplias, destacando la contribución del estudio al estado del arte y sugiriendo futuros caminos para la investigación. Es analítica y reflexiva, ofreciendo un espacio para la evaluación crítica de los resultados. Por contraste, la conclusión resume de manera breve los resultados de la investigación y su significancia, reafirmando las principales contribuciones y hallazgos del estudio de manera concisa sin la profundidad analítica detallada de la discusión. Esencialmente, la discusión profundiza en el "por qué" y el "qué significa" de los hallazgos, mientras que la conclusión proporciona una encapsulación clara y breve de lo que el estudio logró y por qué es importante.




