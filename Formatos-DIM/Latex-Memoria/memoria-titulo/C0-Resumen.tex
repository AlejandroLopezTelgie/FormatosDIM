\section*{Resumen}

El resumen corresponde a una versión sintetizada, pero completa de todo el trabajo de investigación o proyecto. Incluye la declaración del problema, metodología, principales hallazgos y conclusiones. Su objetivo principal es proporcionar a los lectores una visión clara y concisa del contenido del trabajo, permitiéndoles determinar rápidamente su relevancia para sus propios intereses.

Primero se debe comenzar con una breve descripción de la pregunta de investigación o problema a resolver que aborda el trabajo. Esto prepara el escenario para el lector, destacando la importancia de la investigación y su posible contribución al campo. Se debe asegurar que esta declaración sea específica y directamente relacionada con el alcance del estudio. Esta parte del resumen finaliza estableciendo claramente cuál es el objetivo general del trabajo.

Luego, se deben resumir los métodos que fueron utilizados para llevar a cabo la investigación. Esta parte debe abarcar el diseño de la investigación, técnicas de recolección de datos y métodos de análisis. Esta descripción debe mantenerse breve, pero lo suficientemente informativa para permitir a los lectores entender cómo se realizó la investigación, sin entrar en detalles excesivos.

A continuación, el resumen debe enfocarse en los resultados principales o hallazgos de la investigación. Se deben presentar estos hallazgos de manera sucinta, enfatizando los datos y resultados más significativos. Se debe evitar incluir todos los resultados; en su lugar, seleccionar aquellos que aborden más directamente su pregunta de investigación o problema a resolver, y que tengan las implicaciones más significativas al contexto general.

Finalmente, se incluyen las conclusiones más significativas que se desprenden a partir de los resultados, incluidas sus implicaciones para el estado del arte y la investigación futura potencial. Esta sección debe articular brevemente la importancia de su trabajo y cómo avanza el conocimiento o comprensión dentro de su disciplina.

Un resumen de memoria o tesis debe ser conciso, típicamente no más de una plana de extensión. Se debe escribir de manera clara y directa, evitando jerga y lenguaje complejo para asegurar la accesibilidad a una audiencia amplia. Use preferentemente voz activa y tiempo pretérito perfecto al discutir la investigación realizada, aunque el tiempo presente puede ser apropiado para las implicaciones y conclusiones si se extienden más allá del alcance del trabajo completado.



